% ===================================================================
% c111_fonts.tex -- Final Corrected Logic
% ===================================================================

% 1. CJK SWITCH (may be pre-defined by class)
% -------------------------------------------------------------------
% Only define if not already defined by the document class
% Using LaTeX's \@ifundefined to safely check for conditional
\makeatletter
\@ifundefined{ifcjk}{
    \newif\ifcjk
    \cjkfalse % Default to CJK support DISABLED
}{}
\makeatother
% Note: Options are processed by the document class, not here

% 2. LOAD FONT SPEC AND ALWAYS-ON PACKAGES
% -------------------------------------------------------------------
\RequirePackage{fontspec}
\RequirePackage{microtype}
\RequirePackage{silence}
% 3. CONDITIONAL LOADING OF LANGUAGE & DEPENDENT PACKAGES
% -------------------------------------------------------------------
\ifcjk
% --- CJK MODE ---
\WarningFilter{luatexja}{LuaTeX-ja's patch against the microtype package}

% 1. Load luatexja for Japanese support.
\RequirePackage[pass]{luatexja}
\RequirePackage{luatexja-fontspec, luatexja-ruby}

% 2. Load csquotes immediately after. It will now correctly detect the language.
\RequirePackage{csquotes}

% 3. Set Japanese fonts with HanaMin fallback for rare characters.
% HanaMinA: BMP including CJK Ext-A | HanaMinB: SIP including CJK Ext-B through I
\setmainjfont[
    BoldFont={Source Han Serif Bold},
    AltFont={
        {Range="3400-"4DBF, Font=HanaMinA},    % CJK Unified Ideographs Extension A
        {Range="20000-"3FFFF, Font=HanaMinB}, % SIP: CJK Extensions B-I (rare/historical)
    }
]{Source Han Serif}
\setsansjfont[
    BoldFont={Source Han Sans Bold},
    AltFont={
        {Range="3400-"4DBF, Font=HanaMinA},
        {Range="20000-"3FFFF, Font=HanaMinB},
    }
]{Source Han Sans}
\setmonojfont{HanaMinA}
\else
% --- NON-CJK MODE ---
% 1. Load polyglossia for multi-language support.
\RequirePackage{polyglossia}
\setdefaultlanguage{english}
\setotherlanguages{german,french,russian}

% 2. Load csquotes immediately after. It will now correctly detect the language.
\RequirePackage[autostyle]{csquotes}
\fi

% 4. SET MAIN LATIN FONTS (applies to both modes)
% -------------------------------------------------------------------
\setmainfont{Noto Serif}[
    UprightFont=*-Regular,
    BoldFont=*-Bold,
    ItalicFont=*-Italic,
    BoldItalicFont=*-BoldItalic
]
\setsansfont{Noto Sans}[
    UprightFont=*-Regular,
    BoldFont=*-Bold,
    ItalicFont=*-Italic,
    BoldItalicFont=*-BoldItalic
]
\setmonofont{Source Code Pro}

% 5. MATH FONT SETUP
% -------------------------------------------------------------------
% Math fonts are now set in e_core.sty immediately after unicode-math loads.
% This prevents timing issues with internal unicode-math symbol definitions.

% 6. CONDITIONAL EMPHASIS AND TEXT STYLING
% -------------------------------------------------------------------
\ifcjk
    % CJK emphasis: boten (dots) via luatexja-ruby (already loaded above)
    \ltjsetparameter{kanjiskip={0.1\zw plus 0.04\zw minus 0.04\zw}}

    % Redefine \emph for CJK to use boten
    \let\latinemph\emph
    \renewcommand{\emph}[1]{
        \kenten{#1}%
    }

    % For explicit Latin emphasis within CJK text
    \newcommand{\emphLatin}[1]{\latinemph{#1}}
\fi

% --- End of File ---
