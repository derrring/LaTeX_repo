% TOC
\setcounter{secnumdepth}{3}

% ====================================================================
% == Table of Contents Commands
% ====================================================================

% --- Custom command to print a ToC title ---
% This is your excellent helper command. It contains all the conditional
% logic to choose the right title style.
\newcommand{\printtoctitle}[1]{%
    \ifcsdef{chapter}{%
        % In a book-style class, mimic the chapter title format
        \par\noindent
        \begingroup
        \normalfont\bfseries\large
        %\rule{\columnwidth}{1.5pt}
        \newline
        %\vspace{1ex}
        \MakeUppercase{#1} \par
        \vspace*{.2\baselineskip}
        \noindent\rule{\columnwidth}{0.5pt}
        \endgroup
        \par\nobreak
        \vspace{2em} % Space after the title

    }{%
        % In an article-style class, just use a simple \section*
        \section*{#1}
    }
}

%-----------------------------------------------------------------------
%\etoctocstyle[⟨kind⟩]{⟨number_of_columns⟩}{⟨title⟩}

% part toc color
% \titlecontents provided by titletoc package [deprecated, use etoc instead]
% Define a custom TOC line style for parts
% Customizing the TOC
%\etocsettocstyle{⟨before_toc⟩}{⟨after_toc⟩}
%\etocsetstyle{⟨levelname⟩}{⟨start⟩}{⟨prefix⟩}{⟨contents⟩}{⟨finish⟩}

% --- Command for a two-column, brief ToC ---
\newcommand{\brieftoc}{%
    \begingroup
    \etocsettocstyle{}{\clearpage}%
    \phantomsection  % Create anchor here
    \newgeometry{left=1.5cm, right=1.5cm, top=2.5cm, bottom=2.5cm}
    \ifcsdef{chapter}{
        \setcounter{tocdepth}{1}
        \begin{multicols}{2}
            \printtoctitle{Brief Contents}
            \addcontentsline{toc}{chapter}{Brief Contents}
            \tableofcontents
        \end{multicols}
    }
    {
        \setcounter{tocdepth}{2}
        \begin{multicols}{2}
            \printtoctitle{Brief Contents}
            \addcontentsline{toc}{section}{Brief Contents}
            \tableofcontents
        \end{multicols}
    }

    \restoregeometry
    \setcounter{tocdepth}{3}
    \endgroup
}

% --- Command for a single-column, detailed ToC ---
\newcommand{\detailedtoc}{%
    \begingroup
    \etocsettocstyle{
    }{\clearpage}
    \phantomsection  % Create anchor here

    \ifcsdef{chapter}{
        \setcounter{tocdepth}{3}
        \printtoctitle{Detailed Contents}
        \addcontentsline{toc}{chapter}{Detailed Contents}
    }{
        \setcounter{tocdepth}{3}
        \printtoctitle{Detailed Contents}
        \addcontentsline{toc}{section}{Detailed Contents}

    }
    \tableofcontents
    \endgroup
}

% --- Command for the main ToC ---
\newcommand{\mytoc}{%
    \begingroup
    \etocsettocstyle{\renewcommand{\contentsname}{Contents}}{}
    \phantomsection  % Create anchor here
    \setcounter{tocdepth}{2}
    \printtoctitle{\contentsname}
    \ifcsdef{chapter}{
        \addcontentsline{toc}{chapter}{\contentsname}  % Add this line for book class
    }{
        \addcontentsline{toc}{section}{\contentsname}  % Add this line for article class
    }
    \tableofcontents
    \endgroup
    \clearpage
}

% --- Command to set up a local ToC (e.g., at the start of a chapter) ---

\newcommand{\setuplocaltoc}{%
    \etocsettocstyle{
        \setcounter{tocdepth}{3}
        \par\noindent
        \rule{\linewidth}{1pt}
        \vskip0.5\baselineskip\parindent0em % indentation for all subsequent paragraphs in the group
    } %Code Before
    {\rule{\linewidth}{1pt}\vspace{0.5\baselineskip}} %Code After
}

\setuplocaltoc
%-----------------------------------------------------------------------
%\etoctocstyle[⟨kind⟩]{⟨number_of_columns⟩}{⟨title⟩}

% part toc color
% \titlecontents provided by titletoc package [deprecated, use etoc instead]
% Define a custom TOC line style for parts
% Customizing the TOC
%\etocsettocstyle{⟨before_toc⟩}{⟨after_toc⟩}
%\etocsetstyle{⟨levelname⟩}{⟨start⟩}{⟨prefix⟩}{⟨contents⟩}{⟨finish⟩}

% =====================================================
% COMPREHENSIVE TOC STYLING WITH AUTO-WIDTH ADJUSTMENT
% =====================================================

% Custom dot fill definition
\def\mydotfill{\unskip\nobreak\leaders\hbox{\hss\ . \hss}\hfill\nobreak\relax}

% Define lengths for consistent spacing
\newlength{\tocrightmargin}
\newlength{\tocpartsep}
\newlength{\tocchapternumwidth}
\newlength{\tocsectionnumwidth}
\newlength{\tocsubsectionnumwidth}
\newlength{\tocsubsubsectionnumwidth}
\newlength{\tocnumwidthplus}

% Set fixed lengths
\setlength{\tocrightmargin}{0.5cm}
\setlength{\tocpartsep}{1em}
\setlength{\tocnumwidthplus}{0.5em} % Additional space after numbers

% Additional space after numbers

% =====================================================
% PRE-CALCULATED WIDTH BASED ON PATTERNS
% =====================================================
% Adjust these patterns based on your document structure
% Uses begindocument/end to ensure fonts are fully loaded for accurate width calculations
\AddToHook{begindocument/end}{%
    % For PARTS (Roman numerals up to VIII)
    \settowidth{\@tempdima}{\normalsize VIII\hspace{1em}}%
    \edef\tocpartnumwidth{\the\@tempdima}%
    %
    % For CHAPTERS (up to 99)
    \settowidth{\tocchapternumwidth}{\small 99\hspace{0.5\tocnumwidthplus}}%
    %
    % For SECTIONS - calculate the indentation needed
    \settowidth{\tocsectionnumwidth}{\footnotesize 99.9\hspace{\tocnumwidthplus}}% Include space for alignment
    %
    % For SUBSECTIONS - calculate the indentation needed
    \settowidth{\tocsubsectionnumwidth}{\footnotesize 99.9.9\hspace{\tocnumwidthplus}}%
    %
    % For SUBSUBSECTIONS - calculate the indentation needed
    \settowidth{\tocsubsubsectionnumwidth}{\footnotesize 99.9.9.9\hspace{\tocnumwidthplus}}%
}

% =====================================================
% PART STYLE
% =====================================================
\etocsetstyle{part}
{\begingroup\parindent0pt \nobreak}
{\pagebreak[3]\bigskip}
{\normalsize\rmfamily\centering \color{toccolorpart}%
    \etocifnumbered{
        Part \etocnumber{} --
    }{%
}\MakeUppercase{\etocthename}\par\vspace{\tocpartsep}}
{\endgroup}

% =====================================================
% CHAPTER STYLE
% =====================================================
\etocsetstyle{chapter}
{\begingroup\color{black}}
{\small\parindent0pt }
{\etocifnumbered{%
        \leftskip0em\rightskip\tocrightmargin\parfillskip-\rightskip
        % Print number naturally with consistent space after
        \makebox[\tocchapternumwidth][l]{\etocnumber}\hspace{-0.25em}
        % Calculate hanging indent for wrapped chapter titles
        \hangindent\tocchapternumwidth
        \textsc{\etocthename}\hfill
        \makebox[2.5em][r]{\footnotesize\etocpage}\par
    }{%
        \leftskip0em\rightskip\tocrightmargin\parfillskip-\rightskip
        \textsc{\etocthename}\hfill \footnotesize\etocpage\par
}}
{\endgroup}
% =====================================================
% SECTION STYLE (with improved spacing)
% =====================================================
\etocsetstyle{section}
{\begingroup}
{\footnotesize\parindent0pt}
{\etocifnumbered{%
        \leftskip\tocchapternumwidth
        \rightskip\tocrightmargin\parfillskip-\rightskip
        % Print number naturally with consistent space after
        \makebox[\tocsectionnumwidth][l]{\etocnumber}%%
        % Calculate hanging indent based on actual number width
        \hangindent\tocsectionnumwidth
        \etocname\mydotfill
        \makebox[2.5em][r]{\etocpage}\par
    }{%
        \leftskip\tocchapternumwidth\rightskip\tocrightmargin\parfillskip-\rightskip
        \etocname\hfill \etocpage\par
}}
{\endgroup}

% =====================================================
% SUBSECTION STYLE (with improved spacing)
% =====================================================
\etocsetstyle{subsection}
{\begingroup }
{\footnotesize\rmfamily\mdseries \parindent0pt}
{\etocifnumbered{%
        \leftskip\dimexpr\tocchapternumwidth+\tocsectionnumwidth\relax
        \rightskip\tocrightmargin\parfillskip-\rightskip
        % Print number naturally with consistent space after
        \makebox[\tocsubsectionnumwidth][l]{\etocnumber}%
        % Calculate hanging indent
        \hangindent\tocsubsectionnumwidth
        \etocname\mydotfill
        \makebox[2.5em][r]{\etocpage}\par
    }{%
        \leftskip\dimexpr\tocsectionnumwidth\relax
        \rightskip\tocrightmargin\parfillskip-\rightskip
        \etocname\hfill \etocpage\par
}}
{\endgroup}

% =====================================================
% SUBSUBSECTION STYLE (with improved spacing)
% =====================================================
\etocsetstyle{subsubsection}
{\begingroup}
{\footnotesize\rmfamily\mdseries \parindent0pt}
{\etocifnumbered{%
        \leftskip\dimexpr\tocchapternumwidth+\tocsectionnumwidth+\tocsubsectionnumwidth\relax
        \rightskip\tocrightmargin\parfillskip-\rightskip
        % Print number naturally with consistent space after
        \makebox[\tocsubsubsectionnumwidth][l]{\etocnumber}%
        % Calculate hanging indent
        \hangindent\tocsubsubsectionnumwidth
        \etocname\mydotfill
        \makebox[2.5em][r]{\etocpage}\par
    }{%
        \leftskip\dimexpr\tocchapternumwidth+\tocsectionnumwidth+\tocsubsectionnumwidth\relax
        \rightskip\tocrightmargin\parfillskip-\rightskip
        \etocname\hfill \etocpage\par
}}
{\endgroup}

%======================================================
% APPENDIX STYLE
%======================================================
% ====================================================================
% == APPENDIX STYLING PATCH
% ====================================================================
% This code patches the \appendix command to automatically change the
% ToC style for appendix chapters. It requires the 'etoolbox' package.
% Make sure 'etoolbox' is loaded in your e_core.sty or e_standard_doc.sty
% (It is already loaded via e_core.sty in your setup).

\apptocmd{\appendix}{%
    % Define and apply the new style for appendix chapters in the ToC.
    \etocsetstyle{chapter}
    {\begingroup\color{black}}
    {\small\parindent0pt}
    {%
        \leftskip0em\rightskip\tocrightmargin\parfillskip-\rightskip
        % Define the format for the appendix entry line.
        % We combine "Appendix" with the chapter letter.
        \makebox[\tocchapternumwidth][l]{\appendixname~\etocnumber}\hspace{-0.25em}%
        \hangindent\tocchapternumwidth
        \textsc{\etocthename}\hfill
        \makebox[2.5em][r]{\footnotesize\etocpage}\par
    }
    {\endgroup}%
}{%
    % Code to execute on success (optional, can be left empty)
}{%
    % Code to execute on failure
    \PackageWarning{TOCStyle}{Could not patch the \appendix command!}%
}

% =====================================================
% CONFIGURATION OPTIONS
% =====================================================

% Define toccolorpart if not already defined
\providecolor{toccolorpart}{rgb}{0.2,0.2,0.6} % Default: dark blue

% Optional: Adjust spacing between TOC entries
% \setlength{\etocbeforechapterskip}{0.5em}
% \setlength{\etocbeforesectionskip}{0.3em}
% \setlength{\etocbeforesubsectionskip}{0.2em}

% Optional: Set TOC depth (what levels to include)
% \setcounter{tocdepth}{3} % Include up to subsubsection

% =====================================================
% USAGE NOTES
% =====================================================
% 1. Place this code in your preamble after loading etoc package
% 2. The spacing after section numbers is now consistent (1em) regardless of number width
% 3. You can adjust the spacing by changing \hspace{1em} to:
%    - \hspace{0.8em} for tighter spacing
%    - \hspace{1.2em} for looser spacing
% 4. The code assumes you're using standard LaTeX sectioning commands
% 5. Make sure to define or remove \color{toccolorpart} based on your needs

% =====================================================
% EXAMPLE OF CUSTOM ADJUSTMENTS
% =====================================================
% For different spacing preferences:

% % For tighter number-title spacing:
% Replace all instances of \hspace{1em} with \hspace{0.7em}

% % For wider number-title spacing:
% Replace all instances of \hspace{1em} with \hspace{1.5em}
\endinput
