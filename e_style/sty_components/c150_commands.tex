

% Vertical line
\newcommand\verpar[2]{
    \par\vskip1ex
\noindent\begin{tikzpicture}[
paragraph/.style = {inner sep=0pt,
                    text width=\textwidth-\baselineskip}
                            ]
\node (p1) [paragraph] {#2 \vspace{1pt}};
\path ([xshift=-0.75\baselineskip] p1.south west) coordinate (p2)
            -- node (p3) [sloped] {\color{blue} \footnotesize #1} (p2 |- p1.north);
\draw[very thick] (p2) -- (p3.west) (p3.east) -- (p2 |- p1.north);
\end{tikzpicture}
\vspace{5pt}}

% Marble
\newcommand{\colorcircle}[1]{\tikz\node[circle=3pt, draw=#1, fill=#1, inner sep=1.75pt] (d) at (0,0) {};}
\newcommand{\colorstar}[1]{\tikz\node[star=3pt, draw=#1, fill=#1, inner sep=1.75pt] (d) at (0,0) {};}
\newcommand{\colorcube}[1]{\tikz\node[regular polygon=1pt, draw=#1, fill=#1, inner sep=1.5pt, regular polygon sides=3] (d) at (0,0) {};}
\newcommand{\colordiamond}[1]{\tikz\node[diamond=3pt, draw=#1, fill=#1, inner sep=1.75pt] (d) at (0,0) {};}






% Encircle
\makeatletter
\newcommand*{\encircled}[1]{\relax\ifmmode\mathpalette\@encircled@math{#1}\else\@encircled{#1}\fi}
\newcommand*{\@encircled@math}[2]{\@encircled{$\m@th#1#2$}}
\newcommand*{\@encircled}[1]{%
  \tikz[baseline,anchor=base]{\node[draw,circle,outer sep=0pt,inner sep=.2ex] {#1};}}
\makeatother
