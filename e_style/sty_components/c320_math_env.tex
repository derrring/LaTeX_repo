
%================================================================
%%==================Theorem-like=================================
%================================================================
% offered by \usepackage{amsthm,thmtools}
\let\proof\relax
\let\endproof\relax
\newcommand{\thmbegin}{\raisebox{0.35ex}{$\scriptstyle\blacktriangleright$}\hspace{0.3em}}
\newcommand{\defbegin}{\raisebox{0.2ex}{$\displaystyle\maltese$}\hspace{0.3em}}
%\newcommand{\proofbegin}{\raisebox{0.35ex}{$\scriptstyle\triangleright$}\hspace{0.3em}}
%\newcommand{\exbegin}{\raisebox{0.25ex}{$\textstyle\concavediamond$}\hspace{0.3em}}
\newcommand{\remarkend}{\raisebox{0.35ex}{$\scriptstyle\blacktriangleleft$}\hspace{0.3em}}
\newcommand{\solbegin}{\raisebox{0.1ex}{$-\!\prec$}\hspace{0.3em}}
\newcommand{\solend}{\raisebox{0.1ex}{$\succ\!-$}\hspace{0.3em}}

\declaretheoremstyle[
    spaceabove=7pt, spacebelow=7pt,
    headfont=\normalfont\bfseries,
    notefont=\mdseries, notebraces={(}{)},
    headpunct ={},
    headformat=\NAME~\NUMBER \NOTE,
    bodyfont=\normalfont\itshape,
    postheadspace=1em,
]{general_thm_style}

\declaretheoremstyle[
    spaceabove=7pt, spacebelow=7pt,
    headfont=\normalfont\bfseries,
    notefont=\mdseries, notebraces={(}{)},
    headpunct ={},
    headformat=\NAME~\NUMBER \NOTE,
    bodyfont=\normalfont\itshape,
    postheadspace=1em,
]{def_style}

\declaretheoremstyle[
    spaceabove=7pt, spacebelow=7pt,
    headfont=\normalfont\bfseries,
    notefont=\mdseries, notebraces={(}{)},
    headpunct ={},
    headformat=\NAME~\NUMBER \NOTE,
    bodyfont=\normalfont,
    postheadspace=1em,
]{remark_style}

\declaretheoremstyle[
    spaceabove=7pt, spacebelow=7pt,
    headfont=\normalfont\itshape,
    notefont=\mdseries, notebraces={(}{)},
    bodyfont=\normalfont,
    postheadspace=1em,
    headpunct={.},
    headformat= \NAME \NOTE,
    qed=\qedsymbol,
]{proof_style}

\declaretheoremstyle[
    spaceabove=7pt, spacebelow=7pt,
    headfont=\normalfont,
    notefont=\mdseries, notebraces={(}{)},
    bodyfont=\normalfont,
    postheadspace=1em,
    headpunct={},
    headformat=\solbegin\NAME\solend \NOTE,
    qed=$\blacksquare$
]{solution_style}

\declaretheorem[numbered=no, name=Proof, style=proof_style]{proof}
\declaretheorem[numbered=no, name=Solution, style=solution_style]{solution}
\ifcsdef{chapter}{
    \declaretheorem[numberwithin=chapter, name=Theorem,
    style=general_thm_style]{theorem}
    \declaretheorem[numberwithin=chapter,
    name=Example,style=remark_style]{example}
    \declaretheorem[numberwithin=chapter, name=Exercise,
    style=remark_style]{exercise}
}
{
    \declaretheorem[numberwithin=section, name=Theorem,
    style=general_thm_style]{theorem}
    \declaretheorem[numberwithin=section,
    name=Example,style=remark_style]{example}
    \declaretheorem[numberwithin=section, name=Exercise,
    style=remark_style]{exercise}
}
\declaretheorem[sibling=theorem, name=Proposition,
style=general_thm_style]{proposition}
\declaretheorem[sibling=theorem, name=Definition, style=def_style,]{definition}
\declaretheorem[sibling=theorem, name=Lemma, style=general_thm_style]{lemma}
\declaretheorem[sibling=theorem, name=Corollary,
style=general_thm_style]{corollary}

\declaretheorem[name=Remark,style=remark_style]{remark}
