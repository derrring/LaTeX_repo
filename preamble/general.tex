%======================================================================
%============================LaTex3====================================
%======================================================================
\usepackage{expl3,xparse}
%===========================================================================
% Math stuff
%===========================================================================
\usepackage{amsmath,amssymb,mathtools,empheq}
%amsmath and ntheorem should be loaded before {fontspec}
%mathtools offer pmatirx*
\usepackage{cancel}
%\cancel{} can be used in formulae!\bcancel{}\xcancle{}\cancelto{<value>}{<expression>}
\usepackage{extarrows}
\usepackage{amsthm,thmtools}
\numberwithin{equation}{chapter}
%===========================================================================
%=================== font/encoding configuration ===========================
%===========================================================================
% File encoding
%\usepackage[T1]{fontenc}   %% NOTE: This is now handled by the `fontspec` package

%\usepackage{luaotfload}
\usepackage{luatexja, fontspec, luatexja-fontspec, luatexja-ruby, polyglossia}
%%  {fontspec}: set default non-cjk fonts;
%%  {polyglossia}: set default language; exclude cjk;

\setdefaultlanguage{english}
\setmainfont{Noto Serif}%{Latin Modern Roman} %设置衬线英文字体,
\setsansfont{Noto Sans} %设置无衬线英文字体
\setmonofont[ItalicFont={Source Code Pro}]{Fira Code}%英文等宽字体%Fira Code does not offer italics
\setmainjfont[ItalicFont={FZNewKai-Z03}, BoldFont={Source Han Serif Bold}, BoldItalicFont={Noto Serif}]{Source Han Serif}  %中文字体设置
\setsansjfont{Microsoft YaHei} %设置中文无衬线字体
\setmonojfont{HanaMinA} %设置中文等宽字体字体
%---------------------math font -----------------------
\usepackage[warnings-off={mathtools-colon,mathtools-overbracket}, math-style=ISO,bold-style=ISO]{unicode-math}
%\setmathfont{STIX Two Math}
\setmathfont{STIX Two Math}[StylisticSet={02}] %range={"1D454,"1D488,"1D467}
\setmathfont{TeX Gyre Termes Math}[range={bb, scr, bfscr}]

%\usepackage{microtype} %luatexja does not support microtype

% Meta
%
% Document information
%
\newcommand{\myTitle}{%
  Algebra % add \\ for multiline title 
}
\newcommand{\myPlainTitle}{%
  Algebra%
}

\newcommand{\mySubtitle}{Methods and Applications}

\newcommand{\myDissNumber}{---}
\newcommand{\myName}{Jiongyi Wang}
\newcommand{\myAffiliation}{\protect{Keio University}} 
\newcommand{\myLocation}{Tokyo, Japan}
\newcommand{\myTime}{\monthname\ \the\year}

% DOI
\newcommand{\myDOI}{---}

% Hyphenation
\hyphenation{nano-elec-tro-nic}



% Redefine cite command to include space before
% <http://tex.stackexchange.com/questions/11602/>
\let\origcite\cite%
\def\cite#1{\unskip~\origcite{#1}}

%=======================================================================
%====================== layout setting ===============================
%=======================================================================
\usepackage{geometry}
%packageometry save us from calculate parameters that affect each other
\geometry{inner=2cm,outer=2cm,top=2cm,bottom=2cm}
%out margin in (ams)book class
\setlength{\emergencystretch}{1.5em}
\usepackage{graphicx}
%insert external pictures
%usage \includegraphicx[]{}
%parameter in [] : width, height, = ?cm scale = 0.6
\usepackage[normalem]{ulem}
%\uline{}, \uuline{}, \uwave{},\sout{}, \xout{},\dashuline{},\dotuline{},\sout{}(words only)

\usepackage{fancyhdr}
%control headers and footers
\usepackage{pdfpages}

% Colors and customizations
\usepackage[table,x11names,dvipsnames]{xcolor}

\definecolor{hekishoku}{RGB}{0,127,137}
\definecolor{soga}{RGB}{243,244,127}
\definecolor{usukihada}{RGB}{249,241,192}
\definecolor{tokiiro}{RGB}{243,166,150}
\definecolor{codegray}{rgb}{0.5,0.5,0.5}
\definecolor{codepurple}{rgb}{0.58,0,0.82}
\definecolor{mizu}{RGB}{127,204,227}
\definecolor{ao}{RGB}{0,149,217}
\definecolor{kohaku}{RGB}{234,147,10}
\definecolor{daidai}{RGB}{238,120,0}
\definecolor{chaptercolor}{HTML}{1A254B}
\definecolor{darkblue}{HTML}{1A254B}
\definecolor{linkcolor}{HTML}{2B50AA}
\definecolor{citecolor}{HTML}{2B50AA}
\definecolor{linkcolor}{HTML}{2B50AA}
\definecolor{lightlinkcolor}{HTML}{9A8F97}
\definecolor{darklinkcolor}{HTML}{1A254B}
\definecolor{pink}{HTML}{E05F60}
\definecolor{lightblue}{HTML}{A7BED3}
\definecolor{red}{HTML}{F2545B}
\definecolor{blue}{HTML}{2b50aa}
\colorlet{bg_minted}{LemonChiffon1!50!white}

% Hyperlinks
\usepackage[unicode]{hyperref}
\hypersetup{
    colorlinks=true,
    linkcolor=DodgerBlue3,
    citecolor=red,
    urlcolor=DodgerBlue3,
    anchorcolor=DodgerBlue3,
    filecolor=DodgerBlue3,
    linktoc=page,
}


% Widow and club penalties
\clubpenalty=10000
\widowpenalty=10000
%\displaywidowpenalty=10000

%===========================================================================
% Figures, tables, and captions
%===========================================================================
\usepackage{tabularray}
%\setlength{\extrarowheight}{3pt}
%\newcommand{\tableheadline}[1]{\multicolumn{1}{c}{\spacedlowsmallcaps{#1}}}
\usepackage{caption}
%\captionsetup[figure]{format=plain, font=footnotesize, labelfont=bf}
%\captionsetup[table]{format=plain, font=footnotesize, labelfont=bf}
\usepackage{subcaption}
%\captionsetup[subfigure]{format=hang, font=footnotesize, labelfont=bf, labelsep=period, labelformat=simple,justification=justified,singlelinecheck=false}

\usepackage{adjustbox}
\usepackage{wrapfig}
\usepackage{lscape}
\usepackage{multicol,multirow,booktabs,makecell,array}
\setlength{\abovetopsep}{1ex} %offred extra space above top rule of a table, by booktabs package

\usepackage{titlesec,titletoc,etoc}

% tcolorbox
\usepackage{tcolorbox}
\tcbuselibrary{listings,skins,breakable,fitting,raster,xparse,external}
\tcbset{
	colback=lightgray!15,
	colframe=cyan!40!black,
	fonttitle=\bfseries,breakable,enhanced,
	}
\newtcolorbox[auto counter,number within=chapter,list inside=columns]{special_columns}[2][]{
	colback=lightgray!15,
	colframe=cyan!40!black,
	fonttitle=\bfseries,breakable,enhanced,
	title=Note: #2,#1}

\newcommand{\listofcolumns}{%
	\tcblistof[\chapter*]{columns}{List of Column Boxes}
}


% Code listings
\usepackage{minted}
\setminted{
	autogobble,linenos,breaklines=true, breakanywhere,breaksymbolleft={\mbox{\quad}},breakautoindent=false,breaksymbolindent=0pt,
	bgcolor =bg_minted, %The value of this option must not be a color command. Instead, it must be a color name, given as a string, of a previously-defined color
	fontsize=\scriptsize, 
	frame=lines,framerule=0.75pt,
	numbersep=4pt,
	style=manni
	}
\setmintedinline{
  autogobble,linenos=false, fontsize=\footnotesize,
  bgcolor={},
}


% Algorithms
\usepackage[linesnumbered,lined,ruled]{algorithm2e}
\newcommand\mycommentfont[1]{\scriptsize\ttfamily\textcolor{blue}{#1}}
\SetCommentSty{mycommentfont}



% Graphics
\usepackage{graphicx}
\usepackage{rotating}
\usepackage{tikz,pgf,pgfplots,pgfplotstable}
\usetikzlibrary{shapes, arrows.meta, trees, chains, positioning, shadows, calc,intersections,math,babel,cd,shapes.geometric}
\usepgfplotslibrary{external,groupplots} 


% Vertical line
\newcommand\verpar[2]{
    \par\vskip1ex
\noindent\begin{tikzpicture}[
paragraph/.style = {inner sep=0pt,
                    text width=\textwidth-\baselineskip}
                            ]
\node (p1) [paragraph] {#2 \vspace{1pt}};
\path ([xshift=-0.75\baselineskip] p1.south west) coordinate (p2)
            -- node (p3) [sloped] {\color{blue} \footnotesize #1} (p2 |- p1.north);
\draw[very thick] (p2) -- (p3.west) (p3.east) -- (p2 |- p1.north);
\end{tikzpicture}
\vspace{5pt}}

% Marble
\newcommand{\colorcircle}[1]{\tikz\node[circle=3pt, draw=#1, fill=#1, inner sep=1.75pt] (d) at (0,0) {};}
\newcommand{\colorstar}[1]{\tikz\node[star=3pt, draw=#1, fill=#1, inner sep=1.75pt] (d) at (0,0) {};}
\newcommand{\colorcube}[1]{\tikz\node[regular polygon=1pt, draw=#1, fill=#1, inner sep=1.5pt, regular polygon sides=3] (d) at (0,0) {};}
\newcommand{\colordiamond}[1]{\tikz\node[diamond=3pt, draw=#1, fill=#1, inner sep=1.75pt] (d) at (0,0) {};}



% List items
\usepackage{enumitem}
\setlist{noitemsep}
\setlist[itemize,enumerate,1]{labelindent=\parindent,leftmargin=*}
\setlist[enumerate,1]{label=\fbox{\arabic*.}}





% Page numbers in plain style (chapter titles)

% Adjust distance to footer (default is too large)
\setlength{\footskip}{19pt}

% Chapter font
\newcommand{\chapterNumber}{\fontsize{70}{0}\usefont{OT1}{Alegreya-OsF}{m}{n}}


% TOC
%\renewcommand{\cftpartfont}{\color{chaptercolor}\normalfont}%
%\renewcommand{\cftpartpagefont}{\normalfont}%
\renewcommand{\thepart}{\Roman{part}}

% Chapter number on inside
\titleformat{\part}%command
    [display]%shape
    {\normalfont\centering\large}%format
    {\thispagestyle{empty}\partname~\MakeUppercase{\thepart}}%label
    {1em}%sep
    {\color{darklinkcolor}\MakeUppercase}%beforcode
\titleformat{\chapter}[display]%{<commmand>}[<shape>]
  {\normalfont\bfseries\large} %<format>
  %{\MakeUppercase{\chaptertitlename}}{10pt} %{<label>}{<sep>}
  {\vspace*{-3\baselineskip}\makebox[\linewidth][r]{\color{darklinkcolor}\chapterNumber\thechapter}}{11pt}%{<label>}{<sep>}
  {\raggedright\MakeUppercase} %{<beforcode>}
  [\normalsize\vspace*{.8\baselineskip}\titlerule] %[<aftercode>]

% Chapter abstract
\def\chapterabstract#1{%
  \begingroup
  \baselineskip1.3em
  \leftskip1em
  \rightskip\leftskip\itshape#1
  \par
  \endgroup
}

% Chapter quotes
\usepackage{csquotes} % smart quotes

% Adapted from: <http://tex.stackexchange.com/questions/53377/inspirational-quote-at-start-of-chapter>
\newcommand*\quotefont{\fontfamily{fxl}} % selects Libertine for quote font
\newcommand*\quotesize{60} % if quote size changes, need a way to make shifts relative
% Make commands for the quotes
\newcommand*{\openquote}
    {\tikz[remember picture,overlay,xshift=-4ex,yshift=-2.5ex]
      \node (OQ) {\quotefont\fontsize{\quotesize}{\quotesize}\selectfont\textcolor{Silver}{``}};\kern0pt}

\newcommand*{\closequote}[1]
  {\tikz[remember picture,overlay,xshift=4ex,yshift={#1}]
    \node (CQ) {\quotefont\fontsize{\quotesize}{\quotesize}\selectfont\textcolor{Silver}{''}};}






% CV
\newcommand{\cvleft}[1]{\begin{minipage}[t]{2.5cm}\begin{flushright}#1\end{flushright}\end{minipage}\hspace{5mm}}
\newcommand{\cvright}[1]{\begin{minipage}[t]{8cm}{#1}\end{minipage}}



% Margin notes
\usepackage[fulladjust]{marginnote}
%\setlength{\marginparsep}{5mm}
%\setlength{\marginparwidth}{1in}
%\marginnote instead of \marginpar
\newcommand{\marginnoteteal}[1]{\marginnote{\textcolor{teal}{#1}}}


% Encircle
\makeatletter
\newcommand*{\encircled}[1]{\relax\ifmmode\mathpalette\@encircled@math{#1}\else\@encircled{#1}\fi}
\newcommand*{\@encircled@math}[2]{\@encircled{$\m@th#1#2$}}
\newcommand*{\@encircled}[1]{%
  \tikz[baseline,anchor=base]{\node[draw,circle,outer sep=0pt,inner sep=.2ex] {#1};}}
\makeatother




