
%================================================================
%%==================Theorem-like=================================
%================================================================
% offered by \usepackage{amsthm,thmtools}
\let\proof\relax
\let\endproof\relax
\newcommand{\thmbegin}{\raisebox{0.35ex}{$\scriptstyle\blacktriangleright$}\hspace{0.3em}}
\newcommand{\defbegin}{\raisebox{0.2ex}{$\displaystyle\maltese$}\hspace{0.3em}}
%\newcommand{\proofbegin}{\raisebox{0.35ex}{$\scriptstyle\triangleright$}\hspace{0.3em}}
%\newcommand{\exbegin}{\raisebox{0.25ex}{$\textstyle\concavediamond$}\hspace{0.3em}}
\newcommand{\remarkend}{\raisebox{0.35ex}{$\scriptstyle\blacktriangleleft$}\hspace{0.3em}}

\declaretheoremstyle[
spaceabove=7pt, spacebelow=7pt,
headfont=\normalfont\bfseries,
notefont=\mdseries, notebraces={(}{)},
headpunct ={},
headformat=\thmbegin\NAME~\NUMBER \NOTE,
bodyfont=\normalfont\itshape,
postheadspace=1em,
]{general_thm_style}

\declaretheoremstyle[
spaceabove=7pt, spacebelow=7pt,
headfont=\normalfont\bfseries,
notefont=\mdseries, notebraces={(}{)},
headpunct ={},
headformat=\defbegin\NAME~\NUMBER \NOTE,
bodyfont=\normalfont\itshape,
postheadspace=1em,
]{def_style}

\declaretheoremstyle[
spaceabove=7pt, spacebelow=7pt,
headfont=\normalfont\bfseries,
notefont=\mdseries, notebraces={(}{)},
headpunct ={},
headformat=\NAME~\NUMBER \NOTE,
bodyfont=\normalfont,
postheadspace=1em,
]{remark_style}

\declaretheoremstyle[
spaceabove=7pt, spacebelow=7pt,
headfont=\normalfont\itshape,
notefont=\mdseries, notebraces={(}{)},
bodyfont=\normalfont,
postheadspace=1em,
headpunct={.},
headformat= \NAME \NOTE,
qed=\qedsymbol,
]{proof_style}

\declaretheoremstyle[
spaceabove=7pt, spacebelow=7pt,
headfont=\normalfont\bfseries\itshape,
notefont=\mdseries, notebraces={(}{)},
bodyfont=\normalfont,
postheadspace=1em,
headpunct={:},
headformat=\NAME \NOTE,
qed=$\blacksquare$
]{solution_style}

\declaretheorem[numbered=no, name=Proof, style=proof_style]{proof}
\declaretheorem[numbered=no, name=Solution, style=solution_style]{solution}
\declaretheorem[numberwithin=chapter, name=Theorem, style=general_thm_style]{theorem}
\declaretheorem[sibling=theorem, name=Proposition, style=general_thm_style]{proposition}
\declaretheorem[sibling=theorem, name=Definition, style=def_style]{definition}
\declaretheorem[sibling=theorem, name=Lemma, style=general_thm_style]{lemma}
\declaretheorem[sibling=theorem, name=Corollary, style=general_thm_style]{corollary}
\declaretheorem[numberwithin=chapter, name=Example,style=remark_style]{example}
\declaretheorem[numberwithin=chapter, name=Exercise, style=remark_style]{exercise}
\declaretheorem[name=Remark,style=remark_style]{remark}



%----------  decorate !EXISTED! theorem-like environments ----------------
%------------------------realized by tcolorbox------------------------- 
\tcolorboxenvironment{proof}{% add a vertical bar to the left
    blanker,breakable,left=5mm, before skip=10pt,after skip=10pt, borderline west={1mm}{0pt}{airforceblue!75}}



%----------------------------------------------------------------------
%% Delimiters
%----------------------------------------------------------------------
% offered by \usepackage{mathtools}
\newcommand{\ensemble}[1]{\left\{ #1 \right\}} 		% for braces
\newcommand{\bracks}[1]{\left[#1\right]}		% for brackets
\newcommand{\parens}[1]{\left(#1\right)}		% for parentheses
\newcommand{\innerp}[1]{\left\langle#1\right\rangle}	% for scalar product
% Problematic: the \ensemble{} should allow linebreak within math mode. 
% Potential solution : https://tex.stackexchange.com/questions/78670/how-to-achieve-line-break-in-simple-formula-mode
% as well as https://tex.stackexchange.com/questions/119675/why-allowbreak-is-not-working
% to be used within a \left \right pair to allow line break after comma
\def\middlebreak {\nulldelimiterspace0pt
\right.\allowbreak\mskip 0mu plus .5mu \nulldelimiterspace0pt\left.}%


\newcommand{\bra}[1]{\left\langle#1\right\rvert}		% for bras
\newcommand{\ket}[1]{\left\lvert#1\right\rangle}		% for kets
\newcommand{\braket}[2]{\left\langle#1\middle\vert#2\right\rangle}	% for brakets
\newcommand{\abs}[1]{\left\lvert#1\right\rvert}		% for absolute value
\newcommand{\ceil}[1]{\left\lceil#1\right\rceil}		% for ceiling
\newcommand{\floor}[1]{\left\lfloor#1\right\rfloor}		% for floor
\newcommand{\negpart}[1]{\left[#1\right]_{-}}		% for negative part
\newcommand{\pospart}[1]{\left[#1\right]_{+}}	% for positive part
\newcommand{\norm}[1]{\left\lVert#1\right\rVert}		% for norm
\newcommand{\trinorm}[1]{{\left\vert\kern-0.25ex\left\vert\kern-0.25ex\left\vert #1
    \right\vert\kern-0.25ex\right\vert\kern-0.25ex\right\vert}
    } % for triple vertical norm

